% - * - coding: utf-8 - * -
% % A small hack for curve hyperref. A slightly better hyperlink to this chapter
% \ chapter * {To the reader} \ phantomlabel {chapter: to_the_reader}%
% \ addcontentsline {toc} {chapter} {To the reader}
\ chapter {To the reader} \ label {chapter: to_the_reader}

\ initial {0.25ex} {0.0ex} {H} { \ kern 0.65ex, despite the fact that} the literature on
The Lisp theme is enough and it is easily accessible, this book has its own niche.
Programming requires understanding the fundamental foundations of the language; for Lisp and
Scheme they are non-trivial things like functions of higher order, objects,
continuations and the like. Their ignorance and lack of understanding blocks your path
in the future, since what was still considered difficult today is tomorrow
norm for an educated person.

To explain the nature of these entities, their origin and varieties
We will have to go deep into the details. There is a saying that the lispers know
value of everything, but do not know the price. This book is also aimed at reducing
This abyss in understanding the language with the help of a detailed study of its semantics,
and also the realization of the various possibilities of Lisp that were invented for his
more than thirty years of history.

Lisp "--- this is a pleasant language, in which many fundamental and non-trivial
things are expressed in a simple way. Together with ~ ML, its strongly typed
a colleague, (almost) devoid of side effects, he is typical
representative of the family of applicative programming languages. Study
concepts on which this family is based will no doubt be useful for
students and computer scientists of our and future years. Based on the idea
\ term {function}, "--- an idea that has been honed over the centuries and refined by mathematics,
"--- applicative languages ​​are present almost everywhere, where there are
calculations, manifesting themselves in various forms: beginning by redirecting flows
in ~ \ UNIX , ending with the language of extensions of the editor ~ Emacs and many others
scripting languages. Using such tools without understanding their main
mechanism "--- combinations" --- like attempts to express thoughts with help
separate words instead of a single sentence. For survival can be enough
a few learned phrases, but for a full life all the power of the language is required.


\ section * {Audience} \ label {pref / sect: audience}

The book is intended for a wide audience of specialists:
\ begin {itemize}
  \ item for graduates of universities and students who study the techniques
        implementation of programming languages; applicative or not,
        interpretation or compilation "--- is not important;

  \ item for programmers on Lisp or Scheme who want to understand more clearly
        nuances and the cost of the structures they use to write
        more efficient and portable programs;

  \ item for all lovers of applicative languages ​​that will be found in ~ this
        a lot of interesting reflections on your favorite topic.
\ end {itemize}


\ section * {Philosophy} \ label {pref / sect: philosophy}

This book is based on the course of lectures read in the Master's program of the University
Pierre and Maria ~ Curie; some parts of the course are also taught in the Polytechnic
school.

The topics considered here usually follow the introductory course of applicative
languages ​​like Lisp, Scheme or ~ ML, since such courses are most often
end with a detailed analysis of the language in question. The purpose of this book is "---
as much as possible cover the topic of semantics of applicative languages ​​and the development of their
interpreters and compilers. There are twelve interpreters and
two compilers (in ~ byte code and in ~ C). It does not cost the party and
object "= oriented model (considered with the example of \ Meroon .) Also,
in contrast to many other books, this does not neglect such essential for
family of Lisp things like reflection, introspection, dynamic code generation
and, of course, the same macros.

Part of this book is inspired by two works: << \ english {Anatomy of ~ Lisp} >>
\ cite {all78}, considering the approaches to the realization of Lisp in the seventies,
and ~ << \ english {Operating System Design: The XINU Approach} >> \ cite {com84}, where
provides all the necessary code without hiding any details of the operation of the operating room
system, which completely convinces the reader of the accuracy of the presentation.

In ~ such a spirit "--- accuracy, not ~ conciseness" --- this book is also written,
the main issue of which is the semantics of applicative languages ​​in general and Scheme
in particular. Investigating a variety of implementations, considering their various aspects,
we learn with maximum accuracy how any such system is constructed. we
we will consider most of the problematic issues that cause divisions in the community;
each of these problems will be studied, the options for its solution "--- implemented,
compared and analyzed. I'll tell you everything I know, so that you, the reader,
never ~ entered a dead end from "~ for lack of information, and, moreover, having
such a foundation of knowledge, could experiment independently
with the concepts under consideration.

That's why you can get in ~ the electronic form of the complete code of all programs,
given in this book (for details see page ~ \ pageref {pref / sect: source}).


\ section * {Structure} \ label {pref / sect: structure}

The book is divided into two parts. The first part begins with the realization of the naive
the Lisp interpreter and examines in essence the semantics of Scheme. Here to us
The accuracy of the narrative is required, so we will refine and
to redefine namespaces in different ways ( \ Lisp 1, \ Lisp 2 and ~ t. \, etc.)
continuations (and control structures associated with them), assignment and
variable data structures. We note that as
language becomes overgrown with possibilities, its definition becomes ever more simple,
approaching ~ $ \ lambda $ "= calculus.
turn it into a denotational, strictly mathematical equivalent.

More than six years of teaching practice convinced me that
the approach of gradual language refinement is necessary for a gentle acquaintance with the topic
the study of languages ​​in general and the denotational semantics of computations in particular
"--- a theme that we can not afford to ignore.

The second part of the book follows a different path. Pursuing the goal of making a naive
the implementation of the denotational interpreter is more efficient, we will touch on the topic
acceleration of interpretation (in advance calculating constant values), and then
we implement this preliminary processing (with the help of precompilation) for our
compiler in ~ bytecode. In this part, the preparation of the program for the execution and
The actual execution is clearly separated, so here we will consider
topics like dynamic calculations ( \ ic {eval}), reflection (environments as objects
first-class, self-interpretation, << tower >> interpreters), semantics
macros. Next, we implement the Scheme translator in a ~ C code.

The book closes with the implementation of an object-oriented system, which
will help us in the implementation of some interpreters and compilers.

As you know, repetition is the "mother of learning." All the interpreters
intentionally written in different styles: naive, object-oriented,
based on closures, denotational and ~ t. \, etc. This will allow us to consider
a lot of techniques used in the implementation of applicative languages. Also this
pushes you to reflect on the differences between them. Understanding these differences
(see ~ table ~ \ ref {pref / table: signatures} with hints) is true
understanding of the language and its realizations. Lisp "--- this is not ~ one of such implementations,
this is a \ emph {family} of dialects, each of which has its own unique set
features that we will consider.

\ begin {table}
\ begin {center} \ begin {tabular} {cl}
Chapter & Prototype                     \\
\ hline
1 & \ ic {(eval exp env)}             \\
2 & \ ic {(eval exp env fenv)}        \\
   & \ ic {(eval exp env fenv denv)}   \\
   & \ ic {(eval exp env denv)}        \\
3 & \ ic {(eval exp env cont)}        \\
4 & \ ic {(eval ersk)}             \\
5 & \ ic {((meaning e) rsk)}        \\
6 & \ ic {((meaning er) sr k)}       \\
   & \ ic {((meaning er tail?) k)}    \\
   & \ ic {((meaning er tail?))}      \\
7 & \ ic {(run (meaning er tail?))} \\
10 & \ ic {(-> C (meaning er))}
\ end {tabular} \ end {center}
\ caption {Prototypes of interpreters and compilers.}
\ label {pref / table: signatures}
\ end {table}

Chapters are more or less independent, occupying about ~ 40 ~ pages; each chapter
has a list of exercises, answers to ~ which can be found at the end of the book. List
literature contains not only historically important books that allow tracking
development of Lisp from the 1960's, but also modern works.


\ section * {Preliminary knowledge} \ label {pref / sect: prereqs}

Though I also hope that the book will be fascinating and informative, but it
not ~ be sure to be easy to read. Some of the things described here can be
to comprehend, only if to apply efforts corresponding to their complexity. Speaking
in the language of courtly novels, some subjects of sighs reveal their true
beauty and charm only when we are courteous, but inexorably storm them;
if their rich and complex inner world is not under constant siege, they
and remain unapproachable.

Learning the essence of programming languages ​​requires the possession of tools like
$ \ lambda $ "= calculus and denotational semantics, although the narrative will be
gently, consistently and logically from one topic to the next, this
will not be able to save you from all the necessary efforts.

You will need some preliminary knowledge about Lisp or Scheme;
in particular, knowledge of about thirty basic functions and the ability to understand recursion
without excessive mental strain. The main language of this book is Scheme
(for a quick overview, see the ~ \ pageref {pref / sect: scheme-summary} page )
and its object-oriented extension Meroon .
will help us in considering some problems of representation and implementation of structures
data.

All the programs mentioned in the book have been tested and do work
in the Scheme interpreter. And ~ for those who learn the material of this book, there will not be ~
make it very difficult to port them anywhere ~!


\ section * {Acknowledgments} \ label {pref / sect: thanks}

I ~ should thank the organizations that provided me with the equipment
(Apple ~ Mac ~ SE / 30, then Sony ~ NEWS ~ 3260, subsequently a diverse PC and
PowerBook) and ~ generally made this book possible: the Polytechnic School,
State Institute of Research in the field of Informatics and Automation
(INRIA \ kern -0.1em), the National Center for Scientific Research (CNRS).

Also, I would like to thank those who helped me with everything they could, in creating
this book. In a special debt to Sophie ~ Anglad, Josie ~ Biron, Kathleen ~ Collaway,
Jerome ~ Scheyoks, Jean-Marie ~ Geffroy, Christian ~ Julien, Jean-Jacques ~ Laccrump,
Michel ~ Lemaitrom, Luke ~ Moreau, Jean-Francois ~ Perrault, Daniel ~ Ribbens,
Bernard ~ Serpette, Manuel ~ Serrano, Pierre ~ Be, and also in front of my muse,
Claire ~ H. { \ fnstyle { \ RaggedRight } \ trnote * {Sophie Anglade, Josy Byron, Kathleen
Callaway, J \ ' er \ ^ ome Chaillox, Jean-Marie Geffroy, Christian Jullien,
Jean-Jacques Lacrampe, Michel Lema { \ ^ \ i } tre, Luc Moreau, Jean-Fran \ c cois
Perrot, Daniel Ribbens, Bernard Serpette, Manuel Serrano, Pierre Weis,
Claire ~ N. "--- \ emph {Note ~ Transl.}}}

Of course, of course, all the mistakes that, unfortunately, are inevitably present in the text,
are my own.


\ section * {Notation} \ label {pref / sect: notation}

Fragments of the programs will be typed in \ textcd {such a font that undoubtedly
will remind you of old good typewriters. Some words in the code also
will be typed in \ textit {italics} to denote concepts implicit in
the place of these words.

\ indexC * {.->} { \ protect \ is }
\ indexC * {. =} { \ protect \ equals }
The sign { \ is } is read: << has a value >>, and the sign { \ equals } stands for
equivalence, << has the same meaning as ~ and >>. When parsing the computation
expressions after the vertical line, we will write down the environment in which
calculations are carried out. Here is an example illustrating these agreements:

\ begin {code: lisp}
(let ((a (+ b 1)))
  (let ((f (lambda () a)))
    (foo (f) a))) | \ begin {where}
                    \ - b { \ is } 3
                    \ - foo { \ eq } cons
                    \ end {where} |

| | \ eq | (let ((f (lambda () a))) (foo (f) a)) | \ begin {where}
                                            \ - a { \ is } 4
                                            \ - b { \ is } 3
                                            \ - foo { \ eq } cons
                                            \ - f { \ eq } (lambda () a) \ begin {where}
                                                                   \ - a { \ is } 4
                                                                   \ end {where}
                                            \ end {where} |
| | \ eq | (foo (f) a) | \ begin {where}
                  \ - a { \ is } 4
                  \ - b { \ is } 3
                  \ - foo { \ eq } cons
                  \ - f { \ eq } (lambda () a) \ begin {where}
                                         \ - a { \ is } 4
                                         \ end {where}
                  \ end {where} |
| | \ is | (4 .4)
\ end {code: lisp}

All variable names and error messages in ~ the programs that are cited will be
record in English "--- << native language >> Scheme.

We will use several non-standard functions like \ ic {gensym}, which
generates symbols guaranteed not previously encountered in ~ the text of the program.
In the tenth chapter, the functions \ ic {format} and \ ic {pp} will also be used for
formatted output (pretty-printing). These functions are in the majority
Lisp and ~ Scheme implementations.

Some expressions are meaningful only for some of the Lisp dialects like
\ CommonLisp , Dylan, \ EuLisp , \ ISLisp , \ LeLisp , \ footnote * {{ \ LeLisp } is
trademark INRIA.} Scheme and ~ t. \, e. In this case we will write next to
name of the dialect:

\ begin {code: lisp}
(defdynamic fooncall | \ dialect { \ ISLisp } |
  (lambda (one: rest others)
    (funcall one others)))
\ end {code: lisp}

In order to make it easier to navigate in this book, we will use
the { \ seePage *} notation for cross-references to pages. A similar notation
will be used if necessary to point to the exercise: { \ seeEx *}. Also
in the book there is an index with references to all the functions being defined.
\ seePage [chapter: index]


\ section * {Scheme overview} \ label {pref / sect: scheme-summary}

There are many great books for studying Scheme, like \ cite {as85},
\ cite {dyb87}, \ cite {sf89}. We will also rely on the specification described
in the document << The revised and revised revised revised Report on Scheme >>,
whose name is often abbreviated to \ RnRS . ~ \ cite {kcr98}

Now we will only outline the main features of this dialect; those features,
which will then be analyzed in detail as the understanding of the language improves.

In ~ Scheme, you can use symbols, characters, \ trnote * {If there are possible discrepancies,
then the word \ term {sign} will be used in the sense of "printed character"
(character), and the word \ term {symbol} "--- in the usual value for Lisp
(symbol). "--- \ emph {lines}, lists, numbers, booleans,
vectors, ports, and functions (or procedures, as they are commonly called in ~ Scheme).

Each of these data types has a corresponding predicate: \ ic {symbol?},
\ ic {char?}, \ ic {string?}, \ ic {pair?}, \ ic {number?}, \ ic {boolean?}, \ ic {vector?},
\ ic {procedure?}.

In addition to them, there are access procedures and modifiers for those types,
where it makes sense: \ ic {string-ref}, \ ic {string-set!}, \ ic {vector-ref}
and ~ \ ic {vector-set!}.

For lists, they are called \ ic {car}, \ ic {cdr}, \ ic {set-car!} And ~ \ ic {set-cdr!}.

The functions \ ic {car} and \ ic {cdr} can be combined. For example, to access
The second element of the list is \ ic {cadr}.

All values ​​of these types can be written directly to the program.
With symbols and numbers, everything is obvious. Before the signs a prefix is ​​written
\ ic { \ # \ \ bslash }, for example: \ ic { \ # \ bslash Z}, \ ic { \ # \ bslash +},
\ ic { \ # \ bslash space}. The strings are surrounded by \ ic {"} quotes \ ic {"}, lists "---
\ ic {(} with parentheses \ ic {)}. Boolean values ​​are written as \ ic { \ # t}
and \ ic { \ # # f}, respectively. To write vectors, use the syntax
\ ic { \ # (do ~ re ~ mi)}. Naturally, such values ​​can be constructed and
dynamically with the help of \ ic {cons}, \ ic {list}, \ ic {string}, \ ic {make-string},
\ ic {vector}, \ ic {make-vector}. Also in the presence there are casting functions
like \ ic {string-> symbol} and ~ \ ic {int-> char}.

I / O provides the following functions: \ ic {read} reads the input expressions,
\ ic {display} prints them to the screen, and \ ic {newline} goes to the next line.

\ bigskip

\ indexR {form! Scheme concept}
Programs on ~ Scheme are represented by so-called \ term {forms}.

\ indexC {begin}
The form \ ic {begin} allows you to group the forms and compute them sequentially;
for example, \ ic {(begin (display ~ 1) (display ~ 2) (newline))}.

\ indexC {if} \ indexC {cond} \ indexC {else}
\ indexE {Scheme! Boolean values}
\ indexR {Boolean values! in ~ Scheme}
There are several forms of branching. The simplest of these is \ ii {if - then - else},
which is written on ~ Scheme so: \ ic {(if \ ii {condition} \ ii {then}
\ ii {otherwise})}. If there are more than two options, then for this case in ~ Scheme there is
forms \ ic {cond} and ~ \ ic {case}. The ~ \ ic {cond} form contains a list of statements, each
from which begins with the condition "- an expression that returns a logical
value, "--- for which there is a sequence of other forms
(effect). It consistently calculates the conditions of the assertions,
while one of them does not return the truth (or, more accurately, not ~ false, not ~ \ ic { \ # # f}); then
the consequence of this statement is calculated, and the result of its calculation becomes
the result of the whole form \ ic {cond}. Here is an example of using this form, which
at the same time it shows the keyword ~ \ ic {else}:

\ begin {code: lisp}
(cond ((eq? x 'flip)' flop)
      ((eq? x 'flop)' flip)
      (else (list x "neither flip nor flop")))
\ end {code: lisp}

\ indexC {case}
The form \ ic {case} is similar to ~ \ ic {cond}, but it takes the form as the first parameter,
based on the value of which the choice is made between the options. Each of the
The options in the beginning contain a list of values ​​that are suitable for it. how
only a suitable variant is found, it is calculated and this result becomes
the result of the whole form is \ ic {case}. Similarly, at the end there can be a universal
option ~ \ ic {else}. Here's how to rewrite the previous example
with ~ \ ic {case}:

\ begin {code: lisp}
(case x
  ((flip) 'flop)
  ((flop) 'flip)
  (else (list x "neither flip nor flop")))
\ end {code: lisp}

\ indexC {lambda}
\ indexC {let} \ indexC {let *} \ indexC { letrec }
\ indexC {set "!} \ indexC {quote}
Functions are defined by the form ~ \ ic {lambda}. Behind the word ~ \ ic {lambda} follows the list
arguments, and after it "is a sequence of expressions that describe
the actual calculation of a function. The forms ~ \ ic {let}, \ ic {let *} and ~ \ ic {letrec}
define local variables (they differ in the subtleties of the calculation of the initial
values ​​of the defined variables). The values ​​of the variables in the sequel can be
change using the ~ \ ic {set!} form . To write a literal, use
form ~ \ ic {quote}.

\ indexC {define}
\ indexCS {define} {syntax}
\ indexR {syntax! define @ \ protect \ ic {define}}
Using the ~ \ ic {define} form, you can assign a name to any value. She has
special opportunities that we will use. In particular, the possibility
use it as a kind of \ ic {let}, as well as a variant of the syntax of this form,
which makes it easier to define functions. Here's what it means:

\ begin {code: lisp}
(define (rev l)
  (define nil '())
  (define (reverse lr)
    (if (pair? l) (reverse (cdr l) (cons (car l) r)) r))
  (reverse l nil))
\ end {code: lisp}

\ noindent Without syntactic sugar this example looks like this:

\ begin {code: lisp}
(define rev
  (lambda (l)
    (letrec ((reverse (lambda (lr)
                        (if (pair? l) (reverse (cdr l)
                                               (cons (car l) r))
                            r))))
      (reverse l '()))))
\ end {code: lisp}

On this we conclude our overview of Scheme.


\ section * {Source Code} \ label {pref / sect: source}

Programs (interpreted and compiled), given in this book,
object system and tests for them can be collected at the following address:
\ begin {quote}
\ url {http://pagesperso-systeme.lip6.fr/Christian.Queinnec/Books/LiSP-2ndEdition-2006Dec11.tgz}
\ end {quote}

E-mail address of the author of the book:
\ href {mailto: Christian.Queinnec@lip6.fr} { \ nolinkurl {Christian.Queinnec@lip6.fr}}


\ section * {Recommended Reading} \ label {pref / sect: reading}

Since it is implied that you already know Scheme, we will refer to
traditional ~ \ cite {as85, sf89}.

To get more from the book, it makes sense to look into other guides:
\ CommonLisp ~ \ cite {ste90}, Dylan ~ \ cite {app92b}, \ EuLisp ~ \ cite {pe92},
\ ISLisp ~ \ cite {iso94}, \ LeLisp ~ \ cite {cdd + 91}, Oaklisp ~ \ cite {lp88},
Scheme ~ \ cite {kcr98}, T ~ \ cite {ram84}, Talk ~ \ cite {ilo94}.

Finally, for a better understanding of programming languages, in general, it will be useful
book ~ \ cite {bg94}. { \ fnstyle { \ RaggedRight } \ trnote * {In addition, I personally would like to
to advise a wonderful book \ textit {Franklyn ~ Turbak and David ~ Gifford with
Mark ~ A. ~ Sheldon.} Design Concepts in Programming Languages. "--- The MIT Press,
2008. "--- 1352 ~ p." --- \ emph {Note ~ Transl.}}}
